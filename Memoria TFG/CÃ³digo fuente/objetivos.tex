\chapter{Objetivos}
\label{chap:objetivos} 
En este capítulo se explican los objetivos del presente trabajo, la metodología que se ha seguido para alcanzarlos y la planificación que se ha llevado durante el proceso de desarrollo.
   
\section{Objetivos}
En los mundos de los simuladores que se emplean en docencia robótica hay objetos estáticos, móviles y los propios robots programados por los niños. Los objetivos que persiguen este trabajo son los siguientes:

\begin{itemize}
    \item Desarrollar un motor de físicas basado en \textit{A-Frame} que permita replicar de modo realista el movimiento autónomo de los robots programados por los estudiantes de la plataforma \textit{Kibotics} y que se complemente con \textit{CANNON}, el motor por defecto que materializa la gravedad, rozamiento y los choques.
    \item Crear varios ejercicios en la plataforma educativa \textit{Kibotics} que saquen partido del nuevo motor de físicas y sean vistosos, incluyendo sus escenarios y que sirvan de validación experimental.
\end{itemize}

Además, estos objetivos deben satisfacerse cumpliendo los siguientes requisitos:

\begin{itemize}
    \item Materialización de robots con distinta masa y un movimiento autónomo realista, con una aceleración máxima limitada y capacidad de control acotada.
    \item Coexistencia con el motor por defecto \textit{CANNON} que no requiera la modificación de su código fuente.
\end{itemize}

\section{Metodología}
Con el fin de asegurar el correcto desarrollo del Trabajo de Fin de Grado se estableció una reunión semanal con el tutor para compartir los progresos realizados durante la semana y en la que el tutor me pudo orientar sobre dónde dirigir los esfuerzos. Paralelamente a las reuniones semanales, también se ha contado con un canal de slack en el que se encuentran todos los contribuyentes de la plataforma \textit{Kibotics} donde se han podido plantear todo tipo de dudas durante el proceso de aprendizaje. \newline


También se ha elaborado un blog en el que se han ido compartiendo los resultados y el trabajo que se ha realizado cada semana. El blog se ha implementado gracias al dominio gratuito que ofrece GitHub para crear un blog\footnote{https://roboticslaburjc.github.io/2019-tfg-natalia-monforte/} basado en GitHub Pages. En el README de mi Github se ha incluido un enlace pinchable para acceder a dicho blog\footnote{https://github.com/RoboticsLabURJC/2019-tfg-natalia-monforte}. \newline

El modelo de desarrollo software seleccionado ha sido el método en cascada. En primer lugar, se han ido marcando subobjetivos de implementación. Después de finalizar un subobjetivo, se pasaba a una fase de verificación de las funcionalidades y corrección de errores tras la cual se reanudaba el proceso de desarrollo. Las fases en las que se ha dividido la implementación realizada han sido: 

\begin{itemize}
    \item Implementación del movimiento lineal del drone (eje vertical).
    \item Implementación del movimiento lineal para robots terrestres en el plano horizontal.
    \item Implementación de la estabilización del drone cuando se encuentra inmóvil durante el vuelo.
    \item Implementación del movimiento angular para robots terrestres y drones.
\end{itemize}

Para integrar el código de las mejoras o aportaciones realizadas al código fuente de la plataforma \textit{Kibotics}, cabe destacar que se ha utilizado el sistema que ofrece GitHub para integrar código mediante la creación de incidencias, de nuevas ramas y de parches. Para que los desarrolladores pudiesen añadir las nuevas funcionalidades al código fuente oficial de \textit{Kibotics}, se creaba una nueva rama actualizada con los últimos cambios de la rama
principal. Sobre esta rama se desarrollaba la solución a cada incidencia. Una vez incluidos los cambios se explicaban en un comentario o parche y se subían a la nueva rama creada del repositorio de \textit{Kibotics}. El siguiente paso consistía en solicitar la fusión de los cambios de esta rama con
la rama principal, abriendo peticiones \textit{pull request}. Tras la solicitud de la fusión del parche los desarrolladores que cuentan con más experiencia verifican que los cambios son correctos y, si es así, integran los cambios a la rama maestra oficial, dando por resuelta la incidencia. Los comandos necesarios para realizar la integración del código a la rama creada son los siguientes: 

\begin{verbatim}
                git checkout -b issue-XXX
                git add -ruta-del-fichero-a-añadir
                git commit -m "Comentario para el commit"
                git push -u origin issue-XXX
\end{verbatim}

\section{Plan de trabajo}
El proceso de elaboración del Trabajo de Fin de Grado se ha divido en cinco fases distintas:
\begin{itemize}
    \item \textbf{FASE 1:} aprendizaje y primera toma de contacto con las tecnologías web necesarias para la elaboración del trabajo. Especialmente \textit{A-Frame y JavaScript}.
    \item \textbf{FASE 2:} estudio del código de \textit{Kibotics-WebSim} y de las físicas de \textit{A-Frame} (motor de \textit{CANNON}).
    \item \textbf{FASE 3:} Elaboración de un motor de físicas complementario para el simulador \textit{WebSim}.
    \item \textbf{FASE 4:} creación de los primeros mundos utilizando las funcionalidades proporcionadas por \textit{Blender y A-Frame} y creación de nuevos ejercicios para incluir en la plataforma.
\end{itemize}