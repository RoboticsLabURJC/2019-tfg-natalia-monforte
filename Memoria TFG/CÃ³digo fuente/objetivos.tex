\chapter{Objetivos}
\label{chap:objetivos} 
En este capítulo se explican los objetivos del presente trabajo, la metodología que se ha seguido para alcanzar dichos objetivos y la planificación que se ha llevado durante el proceso de investigación.
   
\section{Objetivos}
El principal objetivo de este Trabajo de Fin de Grado es la mejora de las habilidades de desarrollo software mediante diversas actividades de programación. \newline



Con este fin, se han abordado las siguientes tareas:
\begin{itemize}
    \item Exploración del simulador \textit{WebSim} y creación de nuevos mundos para el desarrollo de ejercicios más variados.
    \item Exploración de la tecnología \textit{A-Frame} para la construcción de nuevos escenarios.
    \item Exploración del motor de físicas de \textit{CANNON} para desarrollar unas físicas más realistas para el simulador \textit{WebSim}.
\end{itemize}

\section{Metodología y planificación}
Con el fin de asegurar el correcto desarrollo del Trabajo de Fin de Grado se estableció una reunión semanal con el tutor para compartir los progresos realizados durante la semana y en la que el tutor me pudo orientar sobre dónde dirigir los esfuerzos cada semana. Paralelamente a las reuniones semanales, también se ha contado con un canal de slack en el que se encuentran todos los contibuyentes del entorno \textit{Kibotics} en el que se han podido plantear todo tipo de dudas durante el proceso de aprendizaje. \newline

El proceso de elaboración del Trabajo de Fin de Grado se ha divido en cinco fases distintas:
\begin{itemize}
    \item \textbf{FASE 0:} aprendizaje y primera toma de contacto con las tecnologías web necesarias para la elaboración del trabajo. Especialmente \textit{A-Frame y JavaScript}.
    \item \textbf{FASE 1:} estudio del código de \textit{Kibotics-WebSim}.
    \item \textbf{FASE 2:} creación de los primeros mundos utilizando las funcionalidades proporcionadas por \textit{Blender y A-Frame}.
    \item \textbf{FASE 3:} estudio de las físicas de \textit{A-Frame} (motor de \textit{CANNON}) y elaboración de un motor de físicas complementario para el simulador \textit{WebSim}.
    \item \textbf{FASE 4:} creación de nuevos ejercicios para incluir en la plataforma.
\end{itemize}


También se ha elaborado un blog en el que se han ido compartiendo los resultados y el trabajo que se ha realizado cada semana. El blog se ha implementado gracias al dominio gratuito que ofrece Github para crear un blog\footnote{https://roboticslaburjc.github.io/2019-tfg-natalia-monforte/}. En el README de mi Github se ha incluido un enlace pinchable para acceder a dicho blog\footnote{https://github.com/RoboticsLabURJC/2019-tfg-natalia-monforte}. \newline


Para integrar el código de las mejoras o aportaciones realizadas al código fuente de \textit{Kibotics}, cabe destacar que se ha utilizado el sistema que ofrece Github para integrar código mediante la creación de nuevas ramas y parches. Para que los desarrolladores pudiesen añadir las nuevas funcionalidades al código fuente de \textit{Kibotics}, se creaba una nueva rama actualizada con los últimos cambios de la rama
principal. Sobre esta rama se desarrollaba la solución a cada incidencia. Una vez incluidos los cambios se explicaban en un comentario o commit y se subían a la nueva rama creada del repositorio de \textit{Kibotics}. El siguiente paso consistía en solicitar la fusión de los cambios de esta rama con
la rama principal, abriendo peticiones pull request o parches. Tras la solicitud de la fusión o parche los desarrolladores que cuentan con más experiencia verifican que los cambios son correctos y, si es así, integran los cambios a la rama maestra oficial, dando por resuelta la incidencia. Los comandos necesarios para realizar la integración del código a la rama creada son los siguientes: 

\begin{verbatim}
    git checkout -b issue-XXX
    git add -ruta-del-fichero-a-añadir
    git commit -m "Comentario para el commit"
    git push -u origin issue-XXX
\end{verbatim}

