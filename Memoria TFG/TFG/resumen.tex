\chapter*{Resumen}
\markboth{RESUMEN}{RESUMEN} 
Este trabajo de investigación está enfocado en la mejora del entorno \textit{Kibotics} a partir de la introducción de nuevas funcionalidades basadas en tecnologías web. \textit{Kibotics} es una plataforma de robótica educativa destinada a niños y adolescentes de todas las edades. Este entorno utiliza un simulador llamado \textit{WebSim} para representar en tres dimensiones los escenarios de los ejercicios ofrecidos en la plataforma. Los ejercicios pueden solucionarse tanto en \textit{Scratch} como en \textit{Python} y, además, las soluciones que se desarrollan en el simulador pueden utilizarse posteriormente en los robots físicos. \newline

En particular, este proyecto se va a centrar en la mejora de las físicas de \textit{WebSim} para dotar al simulador de un mayor realismo. Para ello, se va a implementar un nuevo motor complementario que se encargue de materializar la fuerza autónoma de los robots de la escena y que coexista con \textit{CANNON}, el motor por defecto de \textit{A-Frame}, ya que \textit{CANNON} deberá materializar la fricción, la gravedad y las colisiones. Además, también se van a añadir varios novedosos ejercicios a la plataforma con los que se puedan explotar las mejorar que ofrece el nuevo motor de físicas implementado. \newline

La implementación del software del motor de físicas se ha realizado en \textit{JavaScript} en su totalidad. Además, también se ha tratado con el lenguaje \textit{JSON} para modificar los ficheros de configuración de los escenarios de los ejercicios, con \textit{A-Frame} para recrear las escenas de realidad virtual y con \textit{CANNON} por ser el motor de físicas por defecto de \textit{A-Frame}.