\chapter*{Resumen}
\markboth{RESUMEN}{RESUMEN} 
Este Trabajo Fin de Grado está enfocado en la mejora de la plataforma de robótica educativa \textit{Kibotics}. Esta plataforma está dirigida a niños y adolescentes de todas las edades y utiliza un simulador llamado \textit{WebSim}, basado en \textit{A-Frame}, para representar en tres dimensiones los escenarios de los ejercicios ofrecidos. Los ejercicios pueden solucionarse tanto en \textit{Scratch} como en \textit{Python} y, además, las soluciones que se desarrollan en el simulador pueden utilizarse también en los robots físicos. \newline

En particular, este proyecto se centra en la mejora del motor de físicas de \textit{WebSim} para dotar al simulador de un mayor realismo. Para ello, se ha diseñado e implementado un nuevo motor complementario que se encarga de materializar la fuerza autónoma de los robots de la escena y que coexiste con \textit{CANNON}, el motor por defecto de \textit{A-Frame}, ya que \textit{CANNON} materializa la fricción, la gravedad y las colisiones. Además, también se han añadido varios novedosos ejercicios a la plataforma con los que se pueden explotar las mejoras que ofrece el nuevo motor de físicas implementado, incluyendo una aceleración finita realista y sensibilidad a robots con masa grande o pequeña, todo ello configurable. Estos ejercicios han servido de validación experimental.\newline

La implementación del software del motor de físicas se ha realizado en \textit{JavaScript} en su totalidad. Además, también se ha tratado con el lenguaje \textit{JSON} para modificar los ficheros de configuración de los escenarios de los ejercicios.